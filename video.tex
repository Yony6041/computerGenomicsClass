% !TEX root = main.tex  

\section{\href{https://www.youtube.com/watch?v=hS47TcdiKUU}{Beggining and struggles in the age of genomics}.}

\paragraph{Notes about the video:}

\paragraph{\\}
\begin{figure}[htbp]
    \centerline{\includegraphics[width=200mm]{historyGenomics.jpg}}
    \caption{History of Genetics}
    \label{fig4}
\end{figure}

\section{Important years}
1865: Laws of genetics (Mendel laws). \\
1900: Darwin theory + Mendel laws join together \\
1913: Alfred Henry first linear genetic maps \\
1953: Watson and Crick discover the structure of DNA \\
1966: Determination of the genetic code \\
1977: Creating of the first methods to sequence DNA \\
1982: GenBank database was established \\ 
1985: Discovery of the chain reaction polymerase \\
1988: They start to talk about the human genome project \\
1990: The human genome starts and the ELS(Ethical, legal and social implications) \\
1993: The wellcome Trust Sanger Institute \\
1997: Joint Genome institue and National Humans Genome Research Institute (JGI and NHGRI)\\
2003: Finished version of the human genome sequence completed \\

\paragraph{Challenges}
Biology: Concrete the structure and function of the genomes. \\
Health: Find benefits based on the human genome knowledge. \\
Society: Use genomics to maximize the benefis and minimize dangers in society. \\


\begin{itemize}
    \item Humans share 99 of their DNA.
    \item Thymine gets along with Adenine. (hydrogen bonding)
    \item Guanine gets along with Cytosine. (hydrogen bonding)
    \item Model organisms: yeast, bacteria.
\end{itemize}

\paragraph{The biggest challenges: }

\begin{itemize}
    \item Indentify the functional and structural components in the HG.
    \item Understand the organization of the genetic networks and protein routes to see how
    they co-relate to the phenotype in organisms.
    \item To develop a detailed hereditary variation in the HG.
    \item To comprehend the genetic variation between species as a factor for evolution.
    \item To develop laws that allow the generalized use of the information
    about the HG for both research purposes and clinic purposes. 
    \item Find diseases caused by pecific gens.
    \item Use probability to predict genetic diseases and drug side effects.
    \item Kill mosquitos using genetics haha
    \item Become super humans? 
    \item Relationship (gens) - (human behavior).
    \item Find genetic variations that contribute on your helath and resistance against
    diseases.
    \item Ethics boundaries because of course we are mad scientists.
\end{itemize}

\paragraph{The solutions: }

\paragraph*{1. Develop technology}
\begin{itemize}
    \item Sanger sequencing
    \item Marker based genetics.
    \item Cloning.
    \item Chain reaction of the polymerase
\end{itemize}
\paragraph*{2. Make resources available}
\paragraph*{3. Computational Biology}
\paragraph*{4. Education at different levels}
\paragraph*{5. ELSI (Ethic Legal Social implications)}




